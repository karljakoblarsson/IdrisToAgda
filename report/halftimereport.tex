\documentclass[parskip=half]{scrartcl}

\usepackage[style=ieee]{biblatex}
\usepackage{hyperref}
\usepackage{csquotes}
\usepackage{graphicx}
\usepackage[colorinlistoftodos]{todonotes}
% \usepackage{parskip}
% \setlength{\parskip}{10pt}
\usepackage{tikz}
\usetikzlibrary{arrows, decorations.markings}
\usepackage{chngcntr}
\counterwithout{figure}{section}
\usepackage{amsmath}
\usepackage{xcolor}
% \usepackage{listings}

\usepackage{idrislang}

\definecolor{mypink}{RGB}{228,77,77}
\lstset{language=Haskell,keywordstyle={\bfseries \color{mypink}}}
% \lstset{language=Idris,keywordstyle={\bfseries \color{mypink}}}
\lstset{literate={
  {->}{$\rightarrow$}{1}
  {=>}{$\Rightarrow$}{1}
}}
\renewcommand{\IdrisKeyword}[1]{{{\bfseries\color{mypink} #1 }}}

\addbibresource{ita.bib}

\lstdefinelanguage{Lambda}{%
  morekeywords={%
    % if,then,else,fix % keywords go here
    where, data
  },%
  morekeywords={[2]int},   % types go here
  otherkeywords={:}, % operators go here
  literate={% replace strings with symbols
    {->}{{$\rightarrow$}}{2}
    {=>}{{$\Rightarrow$}}{2}
    {lambda}{{$\lambda$}}{1}
    {forall}{{$\forall$}}{1}
  },
  basicstyle={\sffamily},
  keywordstyle={\bfseries},
  keywordstyle={[2]\itshape}, % style for types
  keepspaces,
  mathescape % optional
}[keywords,comments,strings]%

\begin{document}

\begin{titlepage}

\centering
{\scshape\LARGE Master's thesis half-time report}

% * Preliminary title.

\vspace{0.5cm}
{\huge\bfseries Source-to-source translation\\ from Idris to Agda
  }

\vspace{2cm}
{\Large Jakob Larsson\\}
\texttt{<jakob@karljakoblarsson.com>}

\vspace{1.0cm}
{\large Supervisor: Patrik Jansson  \\
        Examiner: Nils Anders Danielsson}

% \vspace{1.5cm}
\vspace{1.5cm}

\vfill
{\large \today}

\end{titlepage}

\tableofcontents
\newpage

%%%%%%%%%%%%%%%%%%%%%%%%%%%%%%%%%%%%%%%%%%%%%%%%%%%%%%%%%%%%%%%%%%%%%%%%%%%%%%%
% Plan for halftime report
% ========================

% Talk about possible verification methods. Pros and cons for each, and how
% feasible they are for use in a MSc thesis. See Wibergh's MSc thesis for
% inspo.

% Talk about differences between Agda and Idris
% Philosphically and technically. And in implementation.

% Problems with how to re-use existing code. Maybe discuss problems with using
% internal interfaces. [TODO gärna med referens till litteraturen]

% Write for someone who is taking the same masters program but may not have
% taken the same electives as you. So someone who has taken, logic in computer
% science, programming language technology and algorithms. But haven't taken AFP
% and types and programming languages. So the basics of a compiler and parser
% is known, and most algorithms used. The reader knows basic functional
% programming and some Haskell. But none of Agda or Idris

% I need to talk about the basics of dependently typed programming, for someone
% with a background in functional programming with algebraic data types (but no
% GADTs) and some logic. So I can just mention Curry-Howard in passing. The
% difference between propositional logic and predicate logic.

% But I need to talk about types parametrizied by values, and not only types.
% And how this relate to theorem proving, difference between Agda/Idris and
% theorem provers such as Coq/Isabelle, a little.  Why is dependently typed
% programming useful? And what are the problems with it?


% Tell about my progress so far:
% ------------------------------
% - Digging in Idris and Agda internals   [bugs found? "bad" design? perhaps refer to Edwin Brady and Idris 2 = Blodwen]
% - Reusing Idris parser and Agda pretty-printer
% - Converting AST-to-AST directly with `ita` [main Idris - to - Agda function]
% - Stats tool, and maybe results from running it

% Short background about Dependently typed programming and the need for it? [Yes - half a page?]
% (leads also to some of the problems: lots of type information needed for the translation
%   challenges in translation "two coupled artefacts": spec + code)

% References:
% -----------
% - Both Idris implementation papers
% - Agda presentation/implementation paper
% - Automatic Agda refactoring thesis maybe? [Definitely]
% - What more? [some SE resource about depencies, legacy code, etc.]

% [perhaps later: Consider what is long term of the learning: not just the code, but the new realisations: advice on moving forward after the project]


% From Canvas:
% ------------

% - In the halftime report you inform your examiner of the current status of your
%   thesis work so she/he can get a picture of the progress.

% - Any significant deviation from the planning report must be stated.

% - The student(s) provide the examiner with a written status report of the
%   project,e.g. a draft of the final reportthat includeswork done so far.
%   This document is usually an extension of the planning report.

% - Often, the written halftime report is combined with a short oral presentation
%   for the examiner (and the supervisor).

%%%%%%%%%%%%%%%%%%%%%%%%%%%%%%%%%%%%%%%%%%%%%%%%%%%%%%%%%%%%%%%%%%%%%%%%%%%%%%%

%%%%%%%%%%%%%%%%
% Comments from NAD. Lite feedback:
% On the planning report.

% * "Agda is mainly focused on automated theorem proving": Snarare
%   "interactive theorem proving".
% Sant, bara att ändra.
%
% * Agda 2.6.0 har nyligen släppts.
% Ja, men det var ju försjutton samma dag som jag bara fokuserade på rapporten.

% * "In Agda there are different levels of Set since the set of types can
%    not be part of itself": Idris använder något liknande:
%     http://docs.idris-lang.org/en/latest/faq/faq.html#does-idris-have-universe-polymorphism-what-is-the-type-of-type
% But it is always infered by the Idris compiler, it can never be specified by
% the user as in Agda.

%
% * "Dependently typed programming is the application of Martin-Löf
%    Intuitionistic Type Theory [3] to practical programming": Det finns
%    andra varianter av typteori.
% Ja, HTT och Cubic tex. Men kanske ändå inpirerat främst av Martin-Löf.
%
% * "However, how to efficiently implement a dependently typed programming
%    language is active research, see for example [6]": Jag undrar varför
%    du valde den referensen. Jag föreslår att du förklarar vad du menar
%    lite tydligare, eller tar bort referensen.
% Bara att ta bort den då.
%
% Är det inget mer jag behöver åtgärda än de kommentarerna är ju allt guld!

%%%%%%%%%%%%%%%%%

%      Todo List
% DONE (For now at least) Have a section about Idris and its major features in
%      the background. I want to talk about mutual blocks, infix etc later.
%      And how those features relate to Agda.

% DONE Use the standard structure in the thesis template.

% DONE Include a updated time plan in the end.

% DONE Look at the refactoring thesis again and try to emulate parts of it.

% DONE Start more general, then zoom in to my topic. YES!!! Finally done!
\section{Introduction}

% Software re-use allows programmers to not repeat themselfs which in theory
% whould increase productivity imensly. However in practice it often never pans
% out. Different technical barriers stop it. Programming language is a big such
% barrier. Libraries or tools writen in one language can not be used with others.
% It is often possible to use foreign function interface (FFI) to interop with
% external code, but it is often cumbersome and hard because different languages
% have different semantics which makes it hard when the conventions don't match.

% The paradigm is actually orthogonal to the type system.
Writing correct software is hard. Over the years several methods and languages
have been developed to make it easier. Types allow the compiler to check some
properties about a program before run-time. More advanced type systems allow
more complicated properties be checked. Dependent types are a advanced type
system in research.

Dependently typed programming languages, however, are far from mainstream.
There are many factors which hinders adoption. One is community fragmentation.
The functional programming community for example is mainly focused on Haskell.
The dependently typed community however, is split between several langauges.
Idris, Agda, ATS, F* and Epigram are some of the most used.

A common obstacle for adoption of a new programming language is the
availability of quality libraries. Productive application development requires
many different libraries.  For example, network protocols, parser combinators
and cryptography.  Implementing all common libraries is a huge task.  There is
a lot of duplicated effort when every language needs to implement similar
libraries.  If it would be possible to re-use libraries between languages,
language implementors could focus on developing their language.
% instead of writing the same libraries over and over for every new language.

Agda~\cite{agda} and Idris~\cite{idris} are two of the most used dependently
typed programming languages.  Agda is mainly focused on interactive theorem
proving while Idris prioritizes general purpose programming.  However, while
they have different focus, both languages share most of their features and
their type systems are similar.

We wish to investigate if it is possible to construct a source-to-source
compiler from Idris to Agda. The source-to-source compiler, from here on called
transpiler, should handle a common subset of Idris and Agda.  The transpiler
would allow Idris libraries to be reused in Agda. This would increase the
number of available Agda libraries and increase the audience for every new
Idris library.

Dependent types allow for all propositions in predicate logic to be encoded as
types. A program which type-checks is the a proof of the proposition encoded in
its type signature. A dependently type programming language is also a theorem
prover. If the transpiler is correct and verified it can be used to increase
the confidence of theorems. A proof which is valid in both languages is
stronger than one which is only valid in one language.
% It will increase the confidence in a proof if it is valid in both languages
% after tranpiling.  It will increase the confidence of a given proof if it is
% valid in both type systems because it is unlikely that both type-checkers
% will have the same bug. This is only valid for a verified transpiler however,
% otherwise there is probably bugs in the transpiler.


% TODO Gather everything about the goals in this section.
% \subsection{Goals}
The goal of this project is to construct a transpiler which can translate
a subset of Idris into valid Agda. The transpiler is intended as an aid to
a human programmer which adapts and Idris library for use in Agda.  It is not
intended as a automatic compiler which runs Idris programs using the Agda
runtime Therfore we will leave untranslatable parts as holes for the user to
manually specify. This also means we aim to produce output which is easy for
the human user to understand.

% Expand
% Since depedently typed programming is often used for formal proofs it is
% important that the user can understand the program.

% The language features which are
% not yet implemented will be left as holes to allow the user to provide the
% missing parts.

% DONE Fix this paragraph
The transpiler should preserve the semantics of the translated program.
Otherwise it would not be a useful tool.  We will verify that the transpiler
preserves the program semantics.  However, this is hard to do formally,
especially with an incomplete transpiler.  The verification will be done as far
as possible.
% Even if it is possible to run both the source and the translated program and
% compare their output, is hard to prove that it is valid for all possible
% inputs.


% TODO START HERE
\subsection{Scope}

It is not feasible to correctly correctly translate between 100\% of the
languages in the scope of a master's thesis.  The languages are not the same,
so we will not work on features only present in Idris or Agda.  Only
transpiling from Idris to Agda will be considered. Bidirectional transpiling is
a interesting goal for further projects.

There will not be any consideration for run-time characteristics of the
translated program. An efficient Idris program may well be translated into an
unusably slow Agda program. The run-time of a dependently typed program is not
a well understood subject. And it depends a lot on the internal working of the
compiler, which can change with every new version.


\section{Background}

% TODO Find relevant articles
A source-to-source compiler, often called a transcompiler or transpiler, is
a compiler which translates source code in one programming language into
another language, usually of similar level of abstraction. A traditional
compiler takes a higher level language as input and outputs a lower level
language, often machine code.  Since Agda and Idris are similar languages we will write a transpiler.

Source-to-source transpiler exists for many language pairs. Most compile
a percived higher power language into a lower power, similar to most
conventional compiler which compile from a high-level language to assembly or
machine code. For example there exists transpilers from Haskell to JavaScript,
even Idris has a JavaScript backend. Transpiling between languages of similar
level has less research.

A transpiler for dependently typed languages has no published attempt.
Dependent types make it harder to construct a transpiled program which
compiles.

% Not useful for the reader.
% Dependently typed programming is the application of
% Intuitionistic Type Theory~\cite{martinlof} to practical programming.

% Dependent types goes further and make it possible to specify propositions from
% predicate logic as types. In theory this allows for the compiler to formally
% verify a program.

Dependent types are types which depend on values. This allows the compiler to
check richer properties about programs than a regular strong static type
system, such as the one in Haskell.
Agda and Idris are two widely used programming languages with support for dependent types.
The Idris website states that Idris is inspired
by Agda but with more focus on practical
programming.\footnote{\url{http://docs.idris-lang.org/en/latest/faq/faq.html\#what-are-the-differences-between-agda-and-idris}}
They are similar in features and in their type-systems.

% TODO Reformulate this paragraph.
Agda is developed as a research language, with the focus on interactive theorem
proving.  An automatic translator from Idris to Agda source code could increase
the usefulness of both languages. The work of implementation of it can give us
greater knowledge of the differences between the type systems and language
features.


% TODO is this a useful heading? I already talk about this two paragraphs up.
\subsection{Dependently Typed Programming}

Dependent types can be seen as an extension of an algebraic type system which
allow types to depend on values.
A common example is a vector type which is indexed on the length of
the vector. This allows the programmer to encode constraints and invariants in
the type system. For example see the signature for \texttt{head} and
\texttt{tail} in listing \ref{lst:depex}.  For \texttt{head} and \textit{tail} to type-check it requires a vector of
length greater than or equal to 1, and can therefore never fail.
The compiler also guarantees that the implementation of the functions satisfy
the properties encoded in the types, otherwise they will not type-check.
This mixing of values
and types goes both ways, types can appear in expression and values can appear
in type signatures.

%%%%%%%%%%%%%%%%%%%%%%%%%%%%%%%%%%%%%%%%
% TODO Improve this section. It is maybe unnecessary, and not relay relevant
% to my thesis.

% Dependent types can be used for proof, see Curry-Howard. But also for providing
% strong correctness guarantees. Which is useful for safety critical systems.

% Dependent types can be seen as the next step in the progression.
% \begin{enumerate}
%   \item Algebraic types
%   \item Parametric Algebraic types
%   \item Generalized Algebraic Data Types
%   \item GADTs
%   \item Full Dependent Types
% \end{enumerate}

% Parametric Algebraic types allow for parametric polymorphism, as used in
% Haskell. This means we can write generic functions like:
% \begin{lstlisting}[language=Lambda]
%     forall a => [a] -> [a] -> [a]
% \end{lstlisting}

% Generalized Algebraic data types allow data constructors whose return values
% are parametrized diffrently. In Haskell syntax:

% \begin{lstlisting}[language=Haskell]
% data Expr a where
%   EBool :: Bool -> Expr Bool
%   EInt :: Int -> Expr Int
% \end{lstlisting}

% Compare to Dependent types where the return type is allowed to depend on both
% the type and the \textit{value} of the type variable.
%%%%%%%%%%%%%%%%%%%%%%%%%%%%%%%%%%%%%%%%

\begin{lstlisting}[language=Haskell,label={lst:depex},caption={ \textit{n} is a type parameter, in this case it is a natural number.  }]
data Vector : Type -> Nat -> Type where
  Nil : {a : Type} -> Vec a 0
  Cons : {a : Type} -> {n : Nat} -> a -> Vector a n -> Vector a (n + 1)

head : Vector a (Suc n) -> a
tail : Vector a (Suc n) -> Vector a n
\end{lstlisting}


% TODO Talk about parametrized types and indexed types, and their differences.
% It will be useful later, when talking about problems with my implementation.
% Maybe in the Idris subsection.


% This part is not necessary I think. It is wrong in several places. And not
% really relevant to my project. Not as a own subsubsection at least.
% \subsubsection{Termination}
% For dependent types to be decidable and consistent every program needs to
% terminate. Otherwise type-checking may take an infinite time.  Both Agda and
% Idris requires programs to terminate by default, but the termination check can
% be bypassed explicitly. This is needed for some programs, especially
% interactive programs, however, it means that the logic is not consistent and
% therefore can prove false statements.



% I should some paper related to transpiling and cite. As well as both Idris and
% Agda implementation-papers. Maybe the faq about differences between the
% languages.

Dependent type systems are still an active research topic and it is hard
to determine if two statements in different systems are equal. This is
a big challenge when trying to show that the transpiler works correctly.
% TODO Specify what I mean, or remove this sentence.
Even if both the source and translated programs type-check, it does not
necessarily mean the translation is correct.

% How to test and verify to translations? It's not feasible to get a lot
% of code runnable without manual intervention. Can I even get it to type-check?
% And what does it mean just that the code is well typed in both languages.
% Equality is hard, especially with dependent types.

\subsection{Verification}

\begin{equation} \label{eq:veri1}
  s_A : A \rightarrow S,
  \ s_I : I \rightarrow S,
\end{equation}
\begin{equation} \label{eq:veri2}
  \forall p\ \text{if}\  s_I(p) == s_A (tr(p))
\end{equation}
% TODO Talk about this equation.
Where $tr(p)$ is the transpiler applied to the program $p$.

% TODO Write more about this.
See equation \ref{eq:veri2}
We don't have either of the functions $s_A$ or $s_I$. They may be possible to
specify, but doing so is far beyond the scope of this project.
It is not possible to enumerate all possible programs $p$.

% TODO Grammar
We need to verify that the transpiler preserves the semantics of the source
program to the transpiled output. However it is hard to do that in general,
especially before the transpiler is complete.  Ideally the transpiled programs
both type-check and runs, then we can compare its outout to that of the source
program.  However, it is hard enough to get the programs to type check without
manual intervention. The type systems are similar but not identical. This means
that even if both the source and translated program type-check we can not be
sure that they represent the same semantics.  Especially if one of the programs
contain holes. It is hard to reason about equality in the context of dependent
types.

\subsection{The Idris programming language}
% TODO Write about Idris and its major features for the programmer. Especially
% how it differs from theory and Agda.
Idris is designed to be similar to Haskell syntactically but with dependent
types. It is specifically aimed at practical programming, compared to most
other dependently typed languages (Agda, Coq, Hol/Isabelle) which are designed
for theorem proving.

% TODO Talk about implicit vs. explicit arguments
One specific difference from Agda relevant for this project is implicit
arguments. In dependently typed programming you often want to have some type as
a parameter to a function, but the compiler can often infer that type, so it is
very redundant to specify the type in every call to the function if the
compiler can infer it itself. Therfore both Agda and Idris allows for a type
argument to be specified implicit. However the syntax differ, in Agda
a implicit argument needs to be defined in the function definition. In Idris
every name in a type signatur which starts with a lower case letter is assumed
to be implicit.



\section{Methods}
% * Method of accomplishment. How should the work be carried out?
Idris is a large language. We need to build the transpiler in small steps. To
maximise utility of the transpiler we should prioritize the most used features,
and leave more obscure features for later.
% TODO Says who? This should be stated in the introduction
The first and most important part is
the expression language, function definitions, application and type
declarations.

Usability oriented features are often are implemented as syntactic sugar and
not part of the core language. They are not necessary to implement in a working
transpiler, but are important for it to be a useful tool. There is a trade-off
there we need to make. If the transpiler outputs code on the level of
assembly, just encoded in Agda syntax, even if correct, it would not be a useful
tool.
% TODO This should be stated in the introduction.
Since this is meant to be a tool to assist a human programmer it must
work on a level where its simple for a human user to understand the generated
code. It is great if it matches the source closely, but that is not an end in
itself.

% We will target Idris 1.3.1 and Agda 2.6.0.


Therefore we wrote a tool to capture usage statistics of the top level abstract
syntax tree (AST). The tool show which syntactic constructs are most often
used, and therefore should be prioritized. It also gives the user the
incentive to refactor Idris code which uses less-used language features,
which may result in cleaner and more easily understood code.

The first step is to translate simple declarations of types and functions,
without dependent types. A subset roughly corresponding to Simply Typed Lambda
Calculus. Since dependent types are the focus of the project the next step is
to translate them. Starting with application at the type level, and functions
definitions. Continuing with indexed data types.


\subsection{Holes}
% Code which is untranslateable
% should be left as holes, with the source as comments.
% TODO Merge the two following paragraphs
There may be valid Idris code which is impossible to translate to Agda.  Those
parts will be left as holes for the user to fill in a reasonable translation in
that specific case.

Where the translator fails we wish to leave holes for the user to provide the
correct translation.
% TODO Have examples for this. Of hole refinement.
The holes must be constructed so that this is possible
which is not obvious to do.


\subsection{Implementation details}

We reuse the Idris complier front-end and AST for our implementation. This
makes sure that we are able to parse all Idris code, and we don't introduce any
bugs in the parser. It could in theory save a lot of time since we don't need
to write a parser. It is however a lot of extra work to reuse real-world code.
Similarly, we reuse the AST and pretty-printer from Agda. This means that most
of the code we need to write is handling the actual translation.

Both Agda and Idris 1 are written in Haskell.  This allows us to build both as
a single Cabal project using Stack.  This makes it easy to reuse all parts of
both languages implementations if needed.  We use Git and import both projects
as Git submodules. This makes the chages changes we make to the upstream
implementations explicit.  Until now we have only made minor changes, to use
the same version of third party libraries in both Agda and Idris. The use of
Git submodules allows us to easily use newer versions of both languages in the
future.

% TODO Maybe this paragraph or the one below is better written than the one
% above.

% Both projects are imported as git submodules in this projects repo, which means
% all changes we have made, to the official implementations are recorded and
% possible to apply again, on updated version. However, we took care to keep the
% modifications minimal to keep the possibility to use the latests versions in
% the future. But since the code depends on a lot of internal interfaces, it will
% still be hard to keep up.

% We pull in both Idris and Agda sources as git submodules in our project
% repository, this makes it possible to record the exact changes we have done to
% the upstream sources. To make it easy to reproduce and extend. It also allows
% use to pull in newer versions of both projects, and the using git rebase, replay
% our changes on top of the newer versions, and make it much easier to maintain
% our tool for future language versions.

% TODO Is there something good in this paragraph?
% Trying to reuse the Agda AST as well has problems. The good thing is that
% Agda is designed to be able to print it's AST back out in the exact same
% representations as the input. However, this means that the parser, AST and
% pretty print has to keep track of a lot of things, not needed for the semantics
% of the program. This made it hard to reconstruct a valid AST from another
% program, we had to guess what parts of the data structure are meaningful or not
% for our use case. And construct dummy data just to keep the compiler
% happy, while not making any difference for the output. We still felt this was
% a reasonable trade off, since constructing a AST and pretty printer from
% scratch would probably be an equal amount of work, but it would be much more
% work to keep it in synch with newer versions of Agda, so the final tool would
% be less useful.





% This is wrong most of it.
% Working step by step, we first translate only simple programs, then adding
% more and more features. Currently unimplemented features are represented with
% holes so that it is always possible to translate code. The output becomes more
% and more complete as the project progresses. The number of holes left for
% a given project also works as an informal method of verification, less holes
% corresponds to a better translator.

To test the implementation we will use the Sequential Decision Problem (SDP)
implementation in
IdrisLibs\footnote{\url{https://gitlab.pik-potsdam.de/botta/IdrisLibs}} which
is one of the bigger Idris codebases available. This will guide our
implementation by first supporting the features used in the core parts of the
SDP library. And then work to support more and more of the library.



% Method
% ------

Ideally we would define a new intermediate language which covers the union of
Agda and Idris features.  This will make clear in the intermediate AST what
features we support and which are not implemented. It also makes it possible to
do bidirectional transpiling in the future.  We could also use QuickCheck to
generate programs in this intermediate language to test the code generation.
This is not done yet.

% TODO Knasiga sista meningar. Fixa språket.
Language features in the source languages which are not supported in the target
language will either fail the transpiler, as they are not possible to represent
in the intermediate language, or require to be compiled to simpler
language feature before being translated to the intermediate. If this is
done it would be helpful for the target language implementers if they would
want to implement that feature in the future.


\subsection{Statistics tool}
In the course of the project we needed a way to prioritize languages features
to implement in the transpiler. Ideally we want the transpiler to handle
as much real-world code as soon as possible. To guide our implementation we
wrote a tool which parses Idris into its AST and then records which data
constructors are most often used.

% TODO
Then we ran this tool on the Idris Prelude. See the results in the table below:


\begin{center}
  \begin{tabular}{ l l l r }
    Variable reference       & PTerm:   &    PRef            &    38.72 \% \\
    Application              & PTerm:   &    PApp            &    25.71 \% \\
    $\rightarrow$            & PTerm:   &    PPi             &    12.05 \% \\
    Patten clauses           & PClause: &    PClauses        &    7.74  \% \\
    Type declaration         & PTerm:   &    PTy             &    5.70  \% \\
    Constant                 & PTerm:   &    PConstant       &    3.87  \% \\
    Interface implementation & PDecl:   &    PImplementation &    1.58  \% \\
    \\
                             & PTerm:   &    PRewrite        &    0.63  \% \\
                             & PDecl:   &    PDirective      &    0.53  \% \\
    Data declaration         & PTerm:   &    PPair           &    0.53  \% \\
                             & PTerm:   &    Placeholder     &    0.48  \% \\
                             & PTerm:   &    PCase           &    0.35  \% \\
    Let clause               & PTerm:   &    PLet            &    0.32  \% \\
    Data declaration         & PDecl:   &    PData           &    0.27  \% \\
    Interface declaration    & PDecl:   &    PInterface      &    0.23  \% \\
    With-clause              & PClause: &    PWith           &    0.21  \% \\
                             & PTerm:   &    PImpossible     &    0.20  \% \\
    Fixity declaration       & PDecl:   &    PFix            &    0.18  \% \\
                             & PTerm:   &    PType           &    0.18  \% \\
                             & PTerm:   &    PAlternative    &    0.17  \% \\
                             & PTerm:   &    PConstSugar     &    0.17  \% \\
    Lambda function          & PTerm:   &    PLam            &    0.13  \% \\
    Namespace declaration    & PDecl:   &    PNamespace      &    0.03  \% \\
    Record declaration       & PDecl:   &    PRecord         &    0.02  \% \\

  \end{tabular}
\end{center}


% We aim to have implementend ?? of the most widely used constructors when this
% project is done.


\subsection{Verification}
It is not obvious how to verify the correctness of the transpiler, especially
while it is a work in progress. It is not feasible to create a 100\% correct
transpiler in the time of this master`s thesis, so we have to somehow verify an
unfinished transpiler.

Even to verify that a single non-trivial program is transpiled correctly is
hard. Since the two languages have slightly different semantics, even a source
and transpiled program pair which seems to be the same for the human
programmer, might have different results for some edge-case. Some possible
verification methods for as single programs, in rough order of difficulty, is
enumerated below:

\begin{enumerate}
\item The program transpiles without errors.
\item The transpiled program type-checks in Agda.
\item Manually inspect that the type signatures are the same.
\item The transpiled program runs without errors.
\item The transpiled program runs with expected output.
\item The source and transpiled programs both returns the same output for
  automatically generated tests.
\item The source and transpiled programs are formally verified to have the same
  semantics.
\end{enumerate}

% \item The transpiler passes generated tests.
% \item The transpiler is formally proved to be correct.

However, we need to show that our transpiler works for different programs.
That rules out some methods which are too manual.  In a perfect world we would
have test suit with a lot of different programs translated and proven correct in
both languages. But no such test suite exists for Agda and Idris.

% TODO This paragraph.
% Ideally we would run both the source and translated program with randomized
% input to automatically test the they return the same result. However, it is
% probably infeasible to get larger programs to type-check and run without manual
% intervention.  Therefore we will use weaker forms of validation.

% TODO This paragraph.
Hopefully we can use the Agda compiler to test if the generated programs
compiles. We can also use this to guide our transpiler, if the final
program does not compile it needs to be re-done. Or the transpiler needs to be
more explicit in that spot. This however require us to keep the source
positions during the translation. Both the Agda and Idris does this, but the
transpiler does not.

To formally prove correctness of the translation is the ideal verification, but
it is far outside the scope of this project, so we will have to settle for less
rigorous methods. It is the safest way, and what would provide full confidence.
But hopefully we can still provide some degree of confidence in the
translation.


% TODO Should I includes some general theory about Dependently typed programming? [Not more than typed lambda calculus]
% TODO Should I write about things which aren't done yet? [as part of an updated plan, yes]

\section{Results}

We have constructed a transpiler from a subset of Idris to Agda. It handles the
expression language, type signatures and data declarations.  It fails for
implicit arguments, which have to be declared in Agda but not in Idris. The
transpiler re-uses both the Idris parser and the Agda pretty-printer from their
main implementations.  This is good, but it has taken time in working with and
around the existing code.  Real-life implementations make a lot of practical
considerations which complicates the code a lot, and are not relevant for this
project.

% We have developed a tool to translate simple Idris programs into Agda. It works
% for a subset of Idris, namely:

% \begin{enumerate}
%   \item Data declarations
%   \item Simple expressions
%   \item And some more things
% \end{enumerate}

The problem of building both project with the same version of GHC and
dependencies took time. It was a manual process of comparing cabal-files and
trying to find versions of dependencies which matches both. And in some cases
change one project to use a more recent version of a dependency with a updated
interface. Idris also uses a custom build process which need some revision to
work in a new folder structure.

% Idris uses a custom build process which was a pain to use with a different
% folder structure. Since we wanted to pull in both Idris and Agda as `git`
% submodules in our repo we needed to change the folder structure and the build
% process. The was maybe not worth the effort.


We have not yet done any verification beyond manual inspection.  But transpiled
programs type-checks in Idris. It is one verification step.  But the goal is
the have a more thorough verification in the end of the project.

% TODO Maybe write something like this.
% In the beginning the process was to run the translation, then try to load the
% program in Agda and see if it compiles. Then manually fix it so that it
% compiles, then changing the transpiler to do the same thing we just did
% manually.

% Implicit arguments are only known and calculated in the elaboration step of
% Idris compilation. There for it is hard to translate them after only parsing.

% TODO Write this better
The transpiler does not handle implicit arguments yet. Agda and Idris have
different type-checkers.
Agda requires more explicit definitions, implicit arguments have to be defined.
Idris automatically considers all lower case
variables in type signatures to be a implicit variable. Since the implicit
arguments in Idris are elaborated in the Type-checker we can not use the parser
output to reconstruct them.  Only after Idris is compiled to the a lower level intermediate
language called \texttt{tt\_elab} the type-checking and elaboration is run.
% TODO Theese to paragraphs should be fixed and merged and spilt to three maybe

Therefore we need to run the elaboration and then extract the implicit
arguments from the intermediate representation before translating them to
Agda.  We are going to do this in a pre-precessing step before the translation,
we translate the intermediate back into the high level AST Idris. This makes it
possible the then just use the main transpiler.  This has the side effect of
making that translation useful for a Idris automatic refactoring tool. It is
often useful to change between implicit and explicit arguments when developing
a dependently typed program.

% TODO This is stated above in Method
% \subsubsection{Statistics tool and its results}
% We have a tool which loads Idris files and parser them. It then uses the
% Declaration/Term AST to calculate which languages features are used and how often
% they are used. See some results from IdrisLibs SDP below.

% This was used to guide and prioritize the implementation of the main tool. For
% the most part it matches our intuitive guess. But it gives us a better argument
% for that the transpiler is useful, even though it is unfinished.

% \subsubsection{Implementation difficulties}
% As expected we ran it to a lot of non project related difficulties when trying
% to reuse existing Agda and Idris sources.

% TODO Should this be in the report?
% The Idris compiler implementation uses the state monad a lot.
% It is almost like an imperative program, just written in Haskell. The
% often touted benefits of functional programming goes out of the window, but the
% compiler is still happy. It goes to show that functional programming is not
% a miracle cure for bad programs, that is still up to the programmer.

% Something something about the crazy Idris implementation with a big state
% monad used everywhere. It is almost an imperative program,
% just written in Haskell.

% The next version of Idris is developed from scratch in a new project, and some
% of the reasons for that is the current implantation, and the difficulties in
% working with it.

% Just trying to find the internal interface of the Idris parser took a long
% time.

\section{Conclusions}

% Nothing right now. Maybe leave this section out for the halftime report.

\newpage
\section{Updated Time Plan}

\subsubsection*{Technical work}
\begin{itemize}

  \item \textit{\color{gray}}[Past] First runnable Dependent types example.
  \item \textit{\color{gray}}[Past] Decide verification method.
  \item (First week in September) Implementation of Implicit arguments done.
  \item (Third week of Semptember) Final implementation version.
  \item (Last week of Semptember) Final verification of the transpiler.
\end{itemize}

\subsubsection*{Writing}
\begin{itemize}
  \item \textit{\color{gray}}[Past] Planning report.
  \item \textit{\color{gray}}[Past] Content complete draft of half-time report.
  \item \textit{\color{gray}}[Past] Final draft of half-time report to supervisor.
  \item (2019-08-10) Half-time report done.
  \item (First week in September) Content complete draft of Background in Final
    report.
  \item (Second week in September) Content complete draft of Final report.
  \item (Fourth week in September) Complete draft of Final report to supervisor
  \item (Second week in October) Final report.
\end{itemize}

\subsubsection*{Compulsory events}
\begin{itemize}
  \item \textit{\color{gray}}[Past] Industry and career advancement seminar
  \item (September) Writing seminar I
  \item (September) Writing seminar II
  \item (September) Opposition.
  \item (First two weeks in October) Presentation.
\end{itemize}

% Maybe useful things to cite:
% R. Milner "Well-typed programs can't go wrong" (1978)

% ~\cite{coquand1992pattern} % Dependent pattern matching is hard
% ~\cite{{quantitative-type-theory} % Blodwen implementation

% \bibliographystyle{plain}

\newpage
\printbibliography{}

\end{document}

% TODO: check possible half-time presentation
%   https://lists.chalmers.se/mailman/listinfo/proglog
%   https://lists.chalmers.se/mailman/listinfo/fp
