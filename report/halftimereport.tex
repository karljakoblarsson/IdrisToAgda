\documentclass[parskip=half]{scrartcl}

\usepackage[style=ieee]{biblatex}
\usepackage{hyperref}
\usepackage{csquotes}
\usepackage{listings}
\usepackage{graphicx}
\usepackage[colorinlistoftodos]{todonotes}
% \usepackage{parskip}
% \setlength{\parskip}{10pt}
\usepackage{tikz}
\usetikzlibrary{arrows, decorations.markings}
\usepackage{chngcntr}
\counterwithout{figure}{section}
\usepackage{xcolor}
\usepackage{listings}

\lstset{language=Haskell,keywordstyle={\bfseries \color{blue}}}

\addbibresource{ita.bib}

\begin{document}

\begin{titlepage}

\centering
{\scshape\LARGE Master thesis Planning report}

% * Preliminary title.

\vspace{0.5cm}
{\huge\bfseries Source-to-source translation\\ from Idris to Agda
  }

\vspace{2cm}
{\Large Jakob Larsson\\}
\texttt{<jakob@karljakoblarsson.com>}

\vspace{1.0cm}
{\large Supervisor: Patrik Jansson  \\
        Examiner: Nils Anders Danielsson}

% \vspace{1.5cm}
\vspace{1.5cm}

\vfill
{\large \today}

\end{titlepage}

%%%%%%%%%%%%%%%%%%%%%%%%%%%%%%%%%%%%%%%%%%%%%%%%%%%%%%%%%%%%%%%%%%%%%%%%%%%%%%%
% Plan for halftime report
% ========================

% Talk about possible verification methods. Pros and cons for each, and how
% feasible they are for use in a MSc thesis. See Wibergh's MSc thesis for
% inspo.

% Talk about differences between Agda and Idris
% Philosphically and technically. And in implementation.

% Problems with how to re-use existing code. Maybe discuss problems with using
% internal interfaces. [TODO gärna med referens till litteraturen]

% Tell about my progress so far:
% ------------------------------
% - Digging in Idris and Agda internals   [bugs found? "bad" design? perhaps refer to Edwin Brady and Idris 2 = Blodwen]
% - Reusing Idris parser and Agda pretty-printer
% - Converting AST-to-AST directly with `ita` [main Idris - to - Agda function]
% - Stats tool, and maybe results from running it

% Short background about Dependently typed programming and the need for it? [Yes - half a page?]
% (leads also to some of the problems: lots of type information needed for the translation
%   challenges in translation "two coupled artefacts": spec + code)

% References:
% -----------
% - Both Idris implementation papers
% - Agda presentation/implementation paper
% - Automatic Agda refactoring thesis maybe? [Definitely]
% - What more? [some SE resource about depencies, legacy code, etc.]

% [perhaps later: Consider what is long term of the learning: not just the code, but the new realisations: advice on moving forward after the project]


% From Canvas:
% ------------

% - In the halftime report you inform your examiner of the current status of your
%   thesis work so she/he can get a picture of the progress.

% - Any significant deviation from the planning report must be stated.

% - The student(s) provide the examiner with a written status report of the
%   project,e.g. a draft of the final reportthat includeswork done so far.
%   This document is usually an extension of the planning report.

% - Often, the written halftime report is combined with a short oral presentation
%   for the examiner (and the supervisor).

%%%%%%%%%%%%%%%%%%%%%%%%%%%%%%%%%%%%%%%%%%%%%%%%%%%%%%%%%%%%%%%%%%%%%%%%%%%%%%%

%%%%%%%%%%%%%%%%
% Comments from NAD. Lite feedback:
% On the planning report.

% * "Agda is mainly focused on automated theorem proving": Snarare
%   "interactive theorem proving".
% Sant, bara att ändra.
%
% * Agda 2.6.0 har nyligen släppts.
% Ja, men det var ju försjutton samma dag som jag bara fokuserade på rapporten.

% * "In Agda there are different levels of Set since the set of types can
%    not be part of itself": Idris använder något liknande:
%     http://docs.idris-lang.org/en/latest/faq/faq.html#does-idris-have-universe-polymorphism-what-is-the-type-of-type
% But it is always infered by the Idris compiler, it can never be specified by
% the user as in Agda.

%
% * "Dependently typed programming is the application of Martin-Löf
%    Intuitionistic Type Theory [3] to practical programming": Det finns
%    andra varianter av typteori.
% Ja, HTT och Cubic tex. Men kanske ändå inpirerat främst av Martin-Löf.
%
% * "However, how to efficiently implement a dependently typed programming
%    language is active research, see for example [6]": Jag undrar varför
%    du valde den referensen. Jag föreslår att du förklarar vad du menar
%    lite tydligare, eller tar bort referensen.
% Bara att ta bort den då.
%
% Är det inget mer jag behöver åtgärda än de kommentarerna är ju allt guld!

%%%%%%%%%%%%%%%%%


\section{Introduction}

% * Background to the assignment. Why is it relevant?
Agda~\cite{agda} and Idris~\cite{idris} are two dependently typed programming
languages.  Agda is mainly focused on interactive theorem proving while Idris
prioritizes general purpose programming.  However, while both languages have
different focus, they share most of their features. Their type systems are
similar.

% Mostly the same as the proposal. And read a few other dependent types
% paper-intros, and write something similar.


% * Aim for the work. What should be accomplished?
% This part is an okay draft now.
We wish to investigate if it is possible to construct a source to source
translation of a common subset from Idris to Agda. A translation which
preserves semantics. The translation should handle as large subset of Idris as
possible, but it is unreasonable to expect to cover 100\% of the language.
A goal for further projects is to do bidirectional translation between the
languages.

This translator would allow Idris libraries to be reused in Agda. A Idris
program may be more easily proved in Agda.  It will also increase the
confidence in a given proof if it is valid in both languages.  The first
challenge is to find a common subset for which it is possible to translate. The
task is to construct a compiler which handles as much of this subset as
possible.


\section{Limitations}
% * Limitations. What should be left out and why?
% TODO This part needs to be improved.
Only translation from Idris to Agda will be considered in this project. We will
target Idris 1.3.1 and Agda 2.6.0. It is not feasible to translate every
language feature and edge case.  The goal is to cover a large subset of Idris
features, but it is hard to give an exact goal.  The translator will built step
by step, supporting a bigger subset of language features for each step. The
translator should be usable early on and then improved.
% progression, and where it's reasonable to land.

% Code which is untranslateable
There may be valid Idris code which is impossible to translate to Agda. There
may also be code where the correct translation is application dependent. Those
parts will be left as holes for the user to fill in a reasonable translation in
that specific case.
% should be left as holes, with the source as comments.

% Dependent records

% Cummutally Set hierarchy. Type in type in Idris.
% Remove since Idris has different levels of Set as well, only it is always
% interfered and cannot be explicitly specified by the programmer.
% In Agda there are different levels of \texttt{Set} since the set of types can
% not be part of itself. This is useful for proofs, but will not be considered in
% this project.

% Automated testing of program output is probably to hard, I need a simpler
% verification criteria.

% Comment from NAD:  Man kan t ex inspektera typsignaturer manuellt och sedan
% förlita sig på att typcheckaren gör sitt jobb (om typsignaturen är
% tillräckligt precis). TODO
% Här är det lite bra input. Det låter som en rimlig start.

\section{Background}
% TODO Change from "The Problem" to "Background".
% * The formulation of the problem at hand and, the assignment. This should
%   include an extended version of the scientific problem definition and
%   references to knowledge within the area given in the thesis proposal.

Dependently typed programming is the application of
% Martin-Löf
% TODO
% Maybe mention about other forms of type theories. Agda is explicitly based on
% Martin-Löf, it says so in the first paragraph of the documentation.
% "It is an extension of Martin-Löf’s type theory, and is the latest in the
% tradition of languages developed in the programming logic group at Chalmers."
Intuitionistic Type Theory~\cite{martinlof} to practical programming.
Types are allowed to depend on values, which allows the compiler to check much
richer properties about the program than Hindley-Milner~\cite{hindley}~\cite{milner}
style types. However, how to efficiently implement a dependently typed
programming language is active research.

Agda and Idris are two of the mostly widely used dependently type programming
languages, with much in common. The Idris website states that Idris is inspired
by Agda but with more focus on practical
programming.\footnote{\url{http://docs.idris-lang.org/en/latest/faq/faq.html\#what-are-the-differences-between-agda-and-idris}}
An automatic translator form Idris to Agda source code could increase the
usefulness of both languages. The work of implementation of it can give us
greater knowledge of the differences between the type systems. As well as the
strengths and weaknesses of both languages.

% Semi-automatic translator is maybe a more correct description.
The goal of this project is to construct a semi-automatic translator which can
translate a subset of Idris into valid Agda. The language features which are
not yet implemented will be left as holes to allow the user to provide the
missing parts.

% I should mention which parts are extra hard. Like: dependent types, records,
% classes/interfaces. Implicit arguments. Maybe partial functions. But I'm not
% sure which parts are yet.

% I should some paper related to transpiling and cite. As well as both Idris and
% Agda implementation-papers. Maybe the faq about differences between the
% languages.

Dependent type systems are still an active research topic, which makes the
translation hard.
% redo this sentence. The question is what it means and not mean if both
% programs type check.
Even if both the source and translated programs type check, it does not
necessarily mean the translation is correct.  Where the translator fails we
wish to leave holes for the user to provide the correct translation. The holes
must be constructed so that is possible which is non-obvious to do.

% How to test and verify to translations? It's not feasible to get a lot
% of code runnable without manual intervention. Can I even get it to type check?
% And what does it mean just that the code is well typed in both languages.
% Equality is hard, especially with dependent types.
It is not obvious how to verify the translation, ideally the translated
programs both type check and can be run. Then it would be possible to compare
the outputs are the same. However it is hard enough to get the programs to type
check without manual intervention. The type systems are similar but not
identical. This means that even if both the source and translated program
type check we can not be sure that they represent the same semantics.
Especially if one of the programs contain holes. It is hard to reason about
equality in the context of dependent types.
% TODO Add specific Ideas about how to verify the implementation.
% Ideas:
% -
% -


\section{Method}
% * Method of accomplishment. How should the work be carried out?
The first step is to translate simple declarations of types and functions,
without dependent types. A subset roughly corresponding to Simply Typed Lambda
Calculus. Since dependent types are the focus of the project the next step is
to translate them. Starting with application in the type level, and functions
definitions. Continuing with indexed data types.
Both Idris and Agda are large languages, some features are more important to
implement then other.  For example \texttt{mutual}-blocks and
\texttt{do}-notation. They are de-sugared in the Idris compiler so are not
strictly necessary to implement. However correctly translating them will
increase the readability and the usefulness of the generated code.
% TODO Talk about the type language.
% type lang
% base types
% variable
% application t1 t2
% function definition (x : t1) -> t2

% Sum/ simple enumeration T | F
% prod/data P = P Bool Bool

% Indexerad data typer

% Kriteriet for att oversatta eller inte ar inte om det ar latt eller
% svart. Both Idris and Agda are large languages. What features are most
% important to translate?

% Developing from the Idris 1 or Blodwen codebase, utilizing as much of the AST
% and pretty printer from the Agda-source as possible.

We wish to reuse the Idris complier front-end for our implementation, by
implementing the translator as a back-end. Hopefully we can reuse the abstract
syntax tree (AST) definition and pretty-printer from the Agda implementation as
well. This means that most of the code needed is handling the actual
translation, minimizing work needed on surrounding infrastructure.  However,
there is always a challenge to interface with existing code.

% | Below is stated elsewhere.
% Another hard task is to find the biggest common subset of the languages.
% It is trivial to name a few common features, bu a useful compiler needs to
% support most widely used features. But there are incompatible differences
% between Agda and Idris which complicates things.

Working step by step, we first translate only very simple programs, then adding
more and more features. Currently unimplemented features are represented with
holes so that it is always possible to translate code. The output becomes more
and more complete as the project progresses. The number of holes left for
a given project also works as an informal method of verification, less holes
corresponds to a better translator.

To test the implementation we will use the Sequential Decision Problem (SDP)
implementation in
IdrisLibs\footnote{\url{https://gitlab.pik-potsdam.de/botta/IdrisLibs}} which
is one of the bigger Idris codebases available. This will guide our
implementation by first supporting the features used in the core parts of the
SDP library. And then work to support more and more of the library.

% Method
% ------

Ideally we will define a new intermediate language which covers the union of
Agda and Idris features. This will make it simpler to translate to and from
both languages. We can also use QuickCheck to generate test programs in this
intermediate language to test the code generation. This is not done yet.
Language features in the source languages which are not supported in the target
language will either fail the translator, as they are not possible to represent
in the intermediate language, or require to be compiled down to simpler
language feature before being translated in to the intermediate. If this is
done it would probably be very helpful for the target language implementors if
they would want to implement that feature in the future.



\subsection{Statistics tool}
In the course of the project we needed a way to prioritize languages features
to implement in the translator. Ideally we want the translator to handle
as much real-world code as soon as possible. To guide our implementation we
wrote a tool which parses Idris into its AST and then records which data
constructors are most often used.

"Future:" Then we ran this tool on the 'IdrisLibs' Sequantial Decision
Problems library. See the results in figure. Right now the translator handles
XX of YY constructors of the Idris Term language and ZZ of WW constructors of
the declaration language.

We aim to have implementend ?? of the most widely used constructors when this
project is done.


\subsection{Verification}
How to verify corectness of the translation is hard. Especially while work is
in progress, because if the translation is not complete the program won't
compile.

\begin{itemize}
\item The first step if is the program compiles.
\item Then if it runs without errors.
\item The next step would be to test the translated program with automatic tests.
  But this is often hard to do fully, and depeneds on the program under test.
\item Finally we could formally prove the program correct and that it is the same
  as the source program. This very much work, and far outside the scope of this
  project.
\end{itemize}

It is also possible to compare the output to hand-translated programs, however,
there are few such programs in Agda and Idris.

In a perfect world we would have test suit with many different programs
translated and proven correct in many languages. To compare our compilers to.
But even then it is not obvious to prove correctness, since programs may look
very different but stil do the same thing.

Ideally we would run both the source and translated program with randomized
input to automatically test the they return the same result. However, it is
probably infeasible to get larger programs to type check and run without manual
intervention.  Therefore we will use weaker forms of validation.

Hopefully we can use the Agda compiler to test if the generated programs
compiles. We can also use this (maybe) to guide are translator, if the final
program does not compile it needs to be redone. Or the translator needs to be
more explicit in that spot. This however require us to keep the source
positions during the translation. Both the Agda and Idris does this, but are
translator does not now.

To formally prove correctness of the translation is the ideal verification, but
it is far outside the scope of this project, so we will have to settle for less
rigorous methods. It is the safest way, and what would provide full confidence.
But hopefully we can still provide some degree of confidence in the
translation.


\section{Results}
% TODO Is this the right section?
% TODO How to structure the report?
% TODO What should I include in the half-time report?
% TODO Should I includes some general theory about Dependently typed programming? [Not more than typed lambda calculus]
% TODO Should I write about things which aren't done yet? [as part of an updated plan, yes]
\subsection{Progress}

We have constructed a translator from a subset of Idris to Agda. It handles the
expression language, type signatures and data declarations.  It fails for
implicit argument, which has to be declared in Agda but not in Idris. The
translator re-uses both the Idris parser and the Agda pretty-printer from their
main implementations. This has several advantages. But also many drawback, much
time has been sunk into working with and most often around the existing code.
Real-life implementations make a lot of practiacal considerations which
complicates the code a lot, and are not relevant for an academic project.

Both projects are imported as git submodules in this projects repo, which means
all changes we have made, to the official implementations are recorded and
possible to apply again, on updated version. However, we took care to make as
few changes as possible to keep the posibillity to use the latests versions in
the future. But since the code depends on a lot of internal interfaces, it will
still be hard to keep up.

Just the problem of building both project with the same version of GHC and
dependencies took some time. It was doable however. Unfourtunally it was a very
manual process of comparing cabal-files and trying to find versions of
dependencies which matches both. And in some cases change one project to use
a more recent version of a dependency with a updated interface.

Implicit arguments are only known and calculated in the elaboration step of
Idris compilation. There for it is hard to translate them after only parsing.

% Something something about the crazy Idris implementation with a big state
% monad used everywhere with a huge state. Its almost an imperative program,
% just written in haskell.



\section{Conclusions}


% Maybe useful things to cite:
% R. Milner "Well-typed programs can't go wrong" (1978)

% ~\cite{coquand1992pattern} % Dependent pattern matching is hard
% ~\cite{{quantitative-type-theory} % Blodwen implementation

% \bibliographystyle{plain}

\printbibliography{}

\end{document}

% TODO: check possible half-time presentation
%   https://lists.chalmers.se/mailman/listinfo/proglog
%   https://lists.chalmers.se/mailman/listinfo/fp
