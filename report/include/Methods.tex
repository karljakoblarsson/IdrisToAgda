% CREATED BY DAVID FRISK, 2016
\chapter{Methods}

The first step is to translate simple declarations of types and functions,
without dependent types. A subset roughly corresponding to Simply Typed Lambda
Calculus. Since dependent types are the focus of the project the next step is
to translate them. Starting with application in the type level, and functions
definitions. Continuing with indexed data types.
Both Idris and Agda are large languages, some features are more important to
implement then other.  For example \texttt{mutual}-blocks and
\texttt{do}-notation. They are de-sugared in the Idris compiler so are not
strictly necessary to implement. However correctly translating them will
increase the readability and the usefulness of the generated code.
% TODO Talk about the type language.
% type lang
% base types
% variable
% application t1 t2
% function definition (x : t1) -> t2

% Sum/ simple enumeration T | F
% prod/data P = P Bool Bool

% Indexerad data typer

% Kriteriet for att oversatta eller inte ar inte om det ar latt eller
% svart. Both Idris and Agda are large languages. What features are most
% important to translate?

% Developing from the Idris 1 or Blodwen codebase, utilizing as much of the AST
% and pretty printer from the Agda-source as possible.

We wish to reuse the Idris complier front-end for our implementation, by
implementing the translator as a back-end. Hopefully we can reuse the abstract
syntax tree (AST) definition and pretty-printer from the Agda implementation as
well. This means that most of the code needed is handling the actual
translation, minimizing work needed on surrounding infrastructure.  However,
there is always a challenge to interface with existing code.

% | Below is stated elsewhere.
% Another hard task is to find the biggest common subset of the languages.
% It is trivial to name a few common features, bu a useful compiler needs to
% support most widely used features. But there are incompatible differences
% between Agda and Idris which complicates things.

Working step by step, we first translate only very simple programs, then adding
more and more features. Currently unimplemented features are represented with
holes so that it is always possible to translate code. The output becomes more
and more complete as the project progresses. The number of holes left for
a given project also works as an informal method of verification, less holes
corresponds to a better translator.

To test the implementation we will use the Sequential Decision Problem (SDP)
implementation in
IdrisLibs\footnote{\url{https://gitlab.pik-potsdam.de/botta/IdrisLibs}} which
is one of the bigger Idris codebases available. This will guide our
implementation by first supporting the features used in the core parts of the
SDP library. And then work to support more and more of the library.
