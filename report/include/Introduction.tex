\chapter{Introduction}
\todo{Determine if parskip or intendation}
\reviewer{You probably want some form of indent in the left margin when you start a new paragraph.
Hard to read without it.}
\todo{Remove all unsupported adjectives}
\reviewer{You use unsupported adjectives (UA) or adjective-like structures, more than you need to.
  I find this is bad style in scientific writing, and they need to be used with care. Use as few as possible.
  ``Delete all the adjectives, and the truth remains.''
  Example: ``far from'', ``plenty'', ``most used''.
  Some can be supported with evidence (``most used'' for example), but then should have evidence attached.
}
\todo{Check with grammarly}

% The paradigm is actually orthogonal to the type system.
Writing correct software is hard. Over the years several methods and languages
have been developed to make it easier. Types allow the compiler to check some
properties about a program before run-time. More advanced type systems allow
more complicated properties be checked.
One advanced types system is dependent types, which is the subject of much
of research.\todo{short cropped sentence}

Dependently typed programming languages, however, are far from mainstream.
There are many factors which hinder adoption. One is community fragmentation.
The community around dependently typed community is split between several
languages.  Idris, Agda, ATS, F* and Epigram are some of the most used.

A common obstacle for adoption of a programming language is the
availability of quality libraries. Productive application development requires
many different libraries.  For example, network protocols, parser combinators
and cryptography.  Implementing all common libraries is a huge task.  There is
a lot of duplicated effort when every language needs to implement similar
libraries.  If it would be possible to re-use libraries between languages,
language implementers could focus on developing their language.
% instead of writing the same libraries over and over for every new language.

% Patrik: Good paragraph
Agda~\cite{agda} and Idris~\cite{idris} are two of the most used dependently
typed programming languages.
The Idris website states that Idris is inspired
by Agda but with more focus on practical
programming.\footnote{\url{http://docs.idris-lang.org/en/latest/faq/faq.html\#what-are-the-differences-between-agda-and-idris}}
Agda's main focus is on interactive theorem proving, though it is used as
a general purpose programming language as well.  However, while they have
different focus, both languages share most of their features and their type
systems are similar.

We wish to investigate if it is possible to construct a source-to-source
compiler, usually called a transpiler, from Idris to Agda.  It should transpile
the common subset of Idris and Agda.  This transpiler would allow Idris
libraries to be reused in Agda. This would increase the number of available
Agda libraries and increase the audience for every new Idris library.

Dependent types allow for all propositions in first order predicate logic to be
encoded as types. A dependently typed program which type-checks is a proof of
the proposition encoded in its type signature. Thus a, dependently type
programming language compiler is also a theorem prover.

A proof system is only useful if it is correct and does not contain bug which
allow faulty proof to pass. Bugs in the system may not be common but they are
severe because a system which allows a faulty proof or rejects a true proof
does not instill confidence.\todo{Redo this sentence}
One use of our transpiler is to increase the confidence of the the theorems it
checks. A proof which is valid in both systems does not depend on bugs present
in only one of the systems.


% Gather everything about the goals in this section.
% \subsection{Goals}
The goal of this project is to construct a transpiler which can translate
a subset of Idris into Agda. The transpiler is intended as an aid for
a human programmer who adapts and Idris library for use in Agda.  It is not
intended as a automatic compiler which runs Idris programs using the Agda
runtime. Therefore we will leave untranslatable parts as holes for the user to
manually fill in. Therefore we aim to produce output which is easy for
the human user to understand.

% Expand
% Since dependently typed programming is often used for formal proofs it is
% important that the user can understand the program.

% The language features which are
% not yet implemented will be left as holes to allow the user to provide the
% missing parts.

The transpiler should preserve the semantics of the translated program.
Otherwise it would not be a useful tool.  We will verify that the transpiler
preserves the program semantics.  However, this is hard to do formally,
especially with an incomplete transpiler.  The verification will be done as far
as possible.
\reviewer{This is strange. Many tools which do not preserve semantics are
useful (say, gcc, which has small bugs). Maybe be more exact here. Also, it is
strange that you say you *will* verify, but that it is hard to do and you will
only do a best-effort approach. It makes sense: verifying a transpiler would be
A LOT of work. Do you even have the Agda and Idris semantics in one or the
other of the tools, or some other prover? Maybe state how you will do it, if
not formally.}

% SOMEWHAT DONE Patrik:
% Could mention [here or later] the relation to refactoring: the translation
% can be seen as divided into two phases where the first phase is an Idris ->
% Idris refactoring [which should be possible to verify within Idris] and the
% second phase is the "refactored Idris" -> Agda part [which is harder to
% verify].

The transpiler could be seen as divided into two passes. First, a Idris to
Idris refactoring to resolve syntactic sugar and gather information from the
type-checker into the AST. The second pass is the Idris to Agda translation.
Verification should be handled sepratly for the two passes.


\subsection{Scope}

\todo{Make this into a list or convert theese statements into flowing
paragraphs.}
It is not feasible to correctly translate all features and constructs availible
in Idris or Agda in the scope of this Master's thesis.

We aim to support features present in both Idris and Agda.
Only transpiling from Idris to Agda will be considered.

\todo{ TODO Skriv vad jag faktiskt hanterar och vad som inte ar gjort och vad
som ar halvags}

There will not be any consideration of runtime performance of the translated
program.  Thus an efficient Idris program may well be translated into an
inefficient Agda program.

We will target Idris 1.3.1 and Agda 2.6.0.

