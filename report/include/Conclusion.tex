% CREATED BY DAVID FRISK, 2016
\chapter{Conclusion}

% You may consider to instead divide this chapter into discussion of the
% results and a summary.

% TODO START HERE
% Write something at all here.

% TODO Write about further research.

% TODO write something about soruce-level transpilation vs. logic level. When
% translating a proof from one system to another it is possible to model features
% of the source system in the target system. But when transpilng source-level
% that is much harder. If I transpiled at the logic level I could have
% implemented universe polymorhpism in Idris. Now I have to rely on the Idris
% built in features. That makes it harder. And makes it novel.

\section{Results}

\todo{This section is only a placeholder for now.}
We have created a tool which translates Idris source code into Agda. The
resulting Agda code can compile and work probably if the Idris source don't use
most language feature.

However Agda and Idris handles universe levels differently so code which relies
on universe cumulativley won't work in Agda. The transpiler doesen't account
for universe levels so int only works on level 1 which is something else.


\section{Discussion}
The utility of transpiler between Agda and Idris is obvious. It has the
possibility of removing a lot of duplicated effort for the developers.
Especially the reuse of libraries and proofs between systems is very valuble.
This is evident from the vast number of previous works in this area, and the
whole reason for projects like Dedukti.~\cite{assaf2016dedukti}

The value of a source-to-source transpiler which produces human readable ouput
is more unclear though. The use case where one wants to continue a started work
in another language is a narrower.

It is certanly possible to construct the transpiler. It only requiers a big
effort before it is a useful tool. Most of the problems are software
engineering problems and not in the logic. But diferences in the foundation
make certain corner cases untranslatable.


\section{Conclusion}
